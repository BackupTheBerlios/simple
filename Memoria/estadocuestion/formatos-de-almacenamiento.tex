\subsection {Los formatos de almacenamiento cl�sicos}

En los primeros sistemas inform�ticos, las capacidades de almacenamiento eran muy limitadas, por lo que ni siquiera se conceb�a la idea de almacenar nada que no fuera lo que conocemos como texto plano. Adem�s, las limitadas capacidades de c�mputo as� como el limitado espacio tanto en discos como en memorias, hac�a totalmente inviable el uso de nada que no fueran caracteres.
A medida que los microprocesadores mejoraban y con el crecimiento del espacio disponible en disco y en memoria, empezaron a aparecer programas que no utilizaban el texto como su formato de almacenamiento sino que empezaron a utilizar archivos binarios, en lugar de los archivos de texto tan populares hasta ese momento. Por ejemplo podemos citar:

\begin {itemize}
\item {Los programas de tratamiento de texto que empezaron a incorporar la posibilidad de incorporar im�genes.}
\item {Los programas de proceso gr�fico que hab�an empezado a hacerse visibles al mercado dom�stico.}
\item {Entornos de desarrollo integrado que permit�an acercar el desarrollo de software al gran p�blico.}
\end {itemize}

El uso de archivos binarios supone que el almacenamiento de informaci�n ya no se hace en texto ``legible'' sino en forma de una secuencia de bits comprensible solamente para el programa que lo cre�. El uso de archivos binarios permiti� que dichos archivos ocuparan menos espacio (por ejemplo, incluyendo un archivo de imagen tal cual, en lugar mediante codificaci�n Base64) y permit�a que los programas trabajaran m�s deprisa, ya que en general, las codificaciones binarias son m�s eficientes.

Se puede decir a grandes rasgos que todos estos programas empezaron lo que supuso una revoluci�n para dos sectores muy claramente diferenciados:

\begin {itemize}

\item {Por un lado, los usuarios vieron como las posibilidades se multiplicaban y se empez� a demandar m�s y m�s. V�ase el caso de los procesadores de texto que permiten no ya la inclusi�n de gr�ficos, sino tambi�n de audio e incluso de v�deo.}
\item {Por otro lado, a la misma velocidad que las posibilidades del usuario crec�an, crec�an los problemas para desarrolladores de aplicaciones que intentaban que los formatos de almacenamiento de sus programas fueran comprensibles para otros programas, o incluso entre distintas versiones. }
\end {itemize}