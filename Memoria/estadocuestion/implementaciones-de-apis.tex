\section {Implementaciones de APIs para XML}

A la hora de manipular documentos XML el W3C deseaba mantener todas las recomendaciones los m�s aisladas posible de ning�n lenguaje de programaci�n o compa��a concreta por lo que decidi� construir un est�ndar denominado DOM (Modelo de Objeto Documento).

El modelo de objeto documento describe una forma clara de manipular ficheros XML de acuerdo a una serie de reglas bien definidas y que respetan los principios de la orientaci�n a objetos. As�, un documento pasa a ser considerado un objeto con una serie de propiedades y m�todos y para el cual se definen una serie de interfaces que permiten su manipulaci�n.

Por ejemplo, en las recomendaciones del W3C se indica que debe existir un objeto que representa un elemento nodo y que se llamar� DOMNode. Adem�s dicho objeto debe tener una serie de m�todos como getChildNode() y que este m�todo devuelve un DOMNode que ser� el nodo hijo del elemento XML con el que se est� tratando. Este elemento DOMNode podr�a implementarse en C++, Java,  Visual Basic o C\#, pero lo verdaderamente importante es que el W3C publica todos los interfaces y cualquier desarrollador puede construir herramientas que implemente el est�ndar DOM con lo que se consigue un doble objetivo:

\begin {itemize}
\item {Por un lado, todos los desarrolladores pueden aprender f�cilmente a manejar nuevas bibliotecas que manipulen ficheros XML ya que conocen los mecanismos de funcionamientos, las clases y los m�todos que ofrecen.}
\item {Los creadores de bibliotecas pueden centrarse en construir aplicaciones eficientes al disponer de un an�lisis y un dise�o claros.}
\end {itemize}

A continuaci�n se describen con m�s detenimiento las principales APIs del W3C

\subsection{El API DOM}

DOM es un Interfaz


\subsection{El API SAX}

SAX es un Interfaz
\subsection{El API SAX}

SAX es un Interfaz
\subsection{Oracle XML-Dev Kit}

El kit de Oracle
\subsection{Xerces}

Xerces es una herramienta XML