\subsection {Los primeros lenguajes de marcas}

A pesar de que los formatos de almacenamiento binarios gozaban de ventajas tales como su reducido tama�o o la eficiencia de su procesamiento, r�pidamente se observ� que los ficheros de texto pose�an otra cualidad muy interesante: Eran legibles por parte de todos los programas con relativa facilidad, e incluso en ciertos casos, comprensible por las personas.

Esta cualidad fue la que llev� al desarrollo de un mecanismo que permitiera almacenar la informaci�n en forma de texto, pero a la vez permitiendo la existencia de datos binarios. Este mecanismo de almacenamiento qued� plasmado en SGML, un lenguaje que defin�a un formato de almacenamiento basado fundamentalmente en texto, pero que inclu�a la definici�n de marcas que daban informaci�n sobre la informaci�n en s�. Esta ``informaci�n sobre la informaci�n'' se conoce con el nombre de metainformaci�n .

SGML significa Standard Generalized Markup Language (Lenguaje de Marcas Generalizado Standard) y fue un serio intento para definir un formato universal para el marcado de informaci�n. SGML adquiri� mucha popularidad entre los desarrolladores de sistemas de gesti�n documentales sin embargo no fue utilizado por los fabricantes de software debido a dos grandes problemas:

\begin {itemize}

\item {Era un lenguaje muy potente, pero como tal, tambi�n supon�a una gran complejidad y supon�a abandonar todo el conocimiento adquirido hasta entonces en cuanto al almacenamiento y empezar a pensar desde una nueva perspectiva.}

\item {Debido a dicha complejidad, el soporte de herramientas para su uso era muy escaso y en ocasiones limitado por lo que en el momento de su aparici�n SGML pr�cticamente generaba m�s problemas de los que resolv�a.}
\end {itemize}

Con el paso del tiempo los formatos binarios ganaron m�s y m�s popularidad hasta la aparici�n de la WWW. Los documentos almacenados en p�ginas Web fueron concebidos para ser ``interpretados'' por un programa espec�fico denominado ``navegador'' o ``browser''. Estos documentos utilizaban una versi�n muy simplificada de SGML que inclu�a marcas para que el navegador supiera de que tama�o ten�a que mostrar un texto o de qu� color. 

El uso de un conjunto de marcas m�s reducido dio lugar a varias consecuencias interesantes para los desarrolladores:

\begin {itemize}
\item {Cualquiera pod�a crear un programa para la creaci�n de p�ginas Web, ya que el conjunto de marcas era peque�o y bien conocido.}
\item {Cualquier p�gina Web pod�a ser modificada por cualquier programa, ya que el an�lisis del texto de las marcas es sencillo.}
\item {La creaci�n de navegadores se hizo relativamente sencilla al simplificarse el lenguaje de marcas.}
\item {Cualquier creador de documentos puede confiar en que su documento se ver� igual en cualquier software de navegaci�n sin necesidad de c�digo adicional.}
\end{itemize}

Hoy d�a, muchos procesadores de texto permiten almacenar los resultados en HTML, lo que muestra la amplia difusi�n de dicho lenguaje de marcas.

